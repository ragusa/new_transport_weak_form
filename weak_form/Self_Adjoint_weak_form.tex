\documentclass{article}
%%%%%%%%%%%%%%%%%%%%%%%%%%%%%%%%%%%%%%%%%%%%%%%%%%%%%%%%%%%%%%%%%%%%%

\usepackage{amssymb}
\usepackage{amsmath}
\usepackage{amsfonts}
\usepackage{amstext}
\usepackage{amsbsy}
\usepackage{mathtools}
\usepackage{xspace}
\usepackage[makeroom]{cancel}

%%%%%%%%%%%%%%%%%%%%%%%%%%%%%%%%%%%%%%%%%%%%%%%%%%%%%%%%%%%%%%%%%%%%
% operators
\renewcommand{\div}{\vec{\nabla}\! \cdot \!}
\newcommand{\grad}{\vec{\nabla}}
% latex shortcuts
\newcommand{\bea}{\begin{eqnarray}}
\newcommand{\eea}{\end{eqnarray}}
\newcommand{\be}{\begin{equation}}
\newcommand{\ee}{\end{equation}}
\newcommand{\bal}{\begin{align}}
\newcommand{\eali}{\end{align}}
\newcommand{\bi}{\begin{itemize}}
\newcommand{\ei}{\end{itemize}}
\newcommand{\ben}{\begin{enumerate}}
\newcommand{\een}{\end{enumerate}}
% shortcut for domain notation
\newcommand{\D}{\mathcal{D}}
% vector shortcuts
\newcommand{\vo}{\vec{\Omega}}
\newcommand{\vx}{\underline{\boldsymbol{x}}}
\newcommand{\vr}{\vec{r}}
\newcommand{\vn}{\vec{n}}
\newcommand{\vnk}{\vec{\mathbf{n}}}
\newcommand{\vj}{\vec{J}}
\newcommand{\me}{\mathcal{E}}
\newcommand{\vme}{\vec{\mathcal{E}}}
\newcommand{\vmeo}{\vec{\mathcal{E}}^o}
\newcommand{\vmf}{\vec{\mathcal{F}}}
\newcommand{\vphi}{\vec{\phi}}
\newcommand{\vphio}{\vec{\Phi}^o}
\newcommand{\vs}{\vec{S}}
% special expressions
\newcommand{\spn}{\ensuremath{\textit{SP}_N}\xspace}
\newcommand{\sn}{\ensuremath{\textit{S}_N}\xspace}
\newcommand{\epsn}{\textit{even-parity} \sn}
\newcommand{\Half}{\text{\it Half\,}}
% jump
\newcommand{\std}{\textit{std}}
\newcommand{\cmp}{\textit{cmp}}
\newcommand{\can}{\textit{can}}
\newcommand{\mi}[1]{\ensuremath{{\textit{#1}}}\xspace}
\newcommand{\upp}[1]{\overline{#1}}
\newcommand{\low}[1]{\underline{#1}}

\newcommand{\direct}{\textit{direct }}
\newcommand{\residual}{\textit{residual }}
\newcommand{\adjoint}{\textit{adjoint }}
% extra space
\newcommand{\qq}{\quad\quad}
% common reference commands
\newcommand{\eqt}[1]{Eq.~(\ref{#1})}                     % equation
\newcommand{\fig}[1]{Fig.~\ref{#1}}                      % figure
\newcommand{\tbl}[1]{Table~\ref{#1}}                     % table
\newcommand{\sct}[1]{Section~\ref{#1}}                   % section
\newcommand{\chp}[1]{Chapter~\ref{#1}}                   % chapter

\newcommand{\mt}[1]{\marginpar{\tiny #1}}
%%%%%%%%%%%%%%%%%%%%%%%%%%%%%%%%%%%%%%%%%%%%%%%%%%%%%%%%%%%%%%%%%%%%%
\begin{document}
%%%%%%%%%%%%%%%%%%%%%%%%%%%%%%%%%%%%%%%%%%%%%%%%%%%%%%%%%%%%%%%%%%%%%


\begin{center}
{\Huge Weak Form for the Self-Adjoint $S_N$ Formalism}
\end{center}

%%%%%%%%%%%%%%%%%%%%%%%%%%%%%%%%%%%%%%%%%%%%%%%%%%%%%%%%%%%%%%%%%%%%%
\section{Self-Adjoint $S_N$}
%%%%%%%%%%%%%%%%%%%%%%%%%%%%%%%%%%%%%%%%%%%%%%%%%%%%%%%%%%%%%%%%%%%%%
\bea
\text{LHS}: &=& (-\vo\cdot\grad + \sigma_a)(\vo\cdot\grad + \sigma_a)\psi \\
            &=& \underbrace{-\vo\cdot\grad\vo\cdot\grad\psi}_{\textcircled{1}} + \underbrace{\left[\sigma_a\vo\cdot\grad\psi - \vo\cdot\grad(\sigma_a\psi)\right]}_{\textcircled{2}} + \underbrace{\sigma_a^2\psi}_{\textcircled{3}} \\
\text{RHS}: &=& (-\vo\cdot\grad + \sigma_a)Q \\
            &=& \underbrace{-\vo\cdot\grad Q}_{\textcircled{4}} + \underbrace{\sigma_a Q}_{\textcircled{5}}
\eea
Note that for radiative transfer problem, $\sigma_a$ is embedded in the source $Q$:
\be
Q = \sigma_a B
\ee
where $B$ is the Planckian source determined by the temperature.


%%%%%%%%%%%%%%%%%%%%%%%%%%%%%%%%%%%%%%%%%%%%%%%%%%%%%%%%%%%%%%%%%%%%%
\section{Weak form}
%%%%%%%%%%%%%%%%%%%%%%%%%%%%%%%%%%%%%%%%%%%%%%%%%%%%%%%%%%%%%%%%%%%%%
Multiply every term in LHS and RHS by basis function $b_i$ and integrate over the problem volume:

LHS:
\begin{align}
 & \boxed{\int\textcircled{1}b_i dV} \nonumber\\
 &= \int -b_i\vo\cdot\grad\vo\cdot\grad\psi dV \\
 &= \underbrace{- \oint(\vo\cdot\grad\psi)\;b_i\;\vo\cdot\vn dA}_{*} + \int (\vo\cdot\grad\psi)(\vo\cdot\grad b_i) dV
\end{align}

\begin{align}
 & \boxed{\int\textcircled{2}b_i dV} \nonumber\\
 &= \int\left[\sigma_a\div(\vo\psi) - \div(\sigma_a\;\vo\psi)\right]\;b_i\;dV \\
 &= \int(-\vo\psi\cdot\grad\sigma_a)\;b_i\;dV = \underline{\underline{\int(-\vo\psi b_i\cdot\grad\sigma_a)dV}} \label{eq:pwl_lhs} \\
 &= \underline{-\oint \sigma_a\;\psi b_i\;\vo\cdot\vn dA + \int\sigma_a\div(\vo\psi b_i) dV} \label{eq:pwc_lhs}
\end{align}

\begin{align}
 & \boxed{\int\textcircled{3}b_i dV} \nonumber\\
 &= \int\sigma_a^2\psi b_i dV
\end{align} 
 
 
RHS:
\begin{align}
 & \boxed{\int\textcircled{4}b_i dV} \nonumber\\
 &= \int (-\vo\cdot\grad Q)b_i dV  \\
 &= \underline{-\oint b_iQ\vo\cdot\vn dA + \int Q\vo\cdot\grad b_i dV} \label{eq:pwc_rhs}\\
 \text{or} \\
 &= -\int\vo\cdot\grad(\sigma_a B)\;b_i\; dV \\
 &= \underline{\underline{-\int\vo(\grad\sigma_a B + \grad B\sigma_a)\;b\;dV}} \label{eq:pwl_rhs}
\end{align}

\begin{align}
 & \boxed{\int\textcircled{5}b_i dV} \nonumber\\
 &= \int\sigma_a\;Q\;b_i\;dV = \int\sigma_a^2\;B\;b_i\;dV
\end{align}

Depending on the order of finite element chosen to represent the solution and material property (opacity (cross-section), source), two different sets of weak form are proposed:

If we choose to use piece-wise constant ($0^{th}$ order) finite element, then it is suggested to use the terms $\underline{underlined}$ whenever possible (as seen in \eqt{eq:pwc_lhs} and \eqt{eq:pwc_rhs}).

If we choose to use piece-wise linear ($1^{st}$ order) finite element, then it is suggested to use the terms $\underline{\underline{double-lined}}$ whenever possible (as seen in \eqt{eq:pwl_lhs} and \eqt{eq:pwl_rhs}).

Note that one can choose to use either the $\underline{underlined}$ or the $\underline{\underline{double-lined}}$, not both. Because they are mathematically equivalent.

%%%%%%%%%%%%%%%%%%%%%%%%%%%%%%%%%%%%%%%%%%%%%%%%%%%%%%%%%%%%%%%%%%%%%
\section{Boundary condition}
%%%%%%%%%%%%%%%%%%%%%%%%%%%%%%%%%%%%%%%%%%%%%%%%%%%%%%%%%%%%%%%%%%%%%
For the in-coming directions, we use Dirichlet boundary condition to explicitly specify the value of $\psi$ on the boundary.

For the out-going directions, we use the first order equation:
\be
\vo\cdot\grad\psi + \sigma_a\psi = Q \Rightarrow \boxed{\vo\cdot\grad\psi = Q - \sigma_a\psi} \label{eq:bc_out}
\ee
The out-going boundary condition \eqt{eq:bc_out} is applied to the $*$ term only, where $\vo\cdot\grad\psi$ is replaced by $Q - \sigma_a\psi$, yielding:
\be
* = \oint(\vo\cdot\grad\psi)\;b_i\;\vo\cdot\vn dA = \underbrace{-\oint Q\;b_i\;\vo\cdot\vn dA}_\text{move to RHS} + \underbrace{\oint\sigma_a\psi b_i \vo\cdot\vn dA}_\text{keep on LHS}, \quad \text{for }\vo\cdot\vn>0
\ee
The $\vo\cdot\vn>0$ portion of this $*$ term is neglected, it is equivalent to force:
\be
* = \oint(\vo\cdot\grad\psi)\;b_i\;\vo\cdot\vn dA = 0, \quad \text{for }\vo\cdot\vn<0
\ee


%%%%%%%%%%%%%%%%%%%%%%%%%%%%%%%%%%%%%%%%%%%%%%%%%%%%%%%%%%%%%%%%%%%%%
\section{Final form after cancellation of the surface terms}
%%%%%%%%%%%%%%%%%%%%%%%%%%%%%%%%%%%%%%%%%%%%%%%%%%%%%%%%%%%%%%%%%%%%%
Take the five kernels listed in the above section, using the $\underline{underlined}$ as seen in \eqt{eq:pwc_lhs} and \eqt{eq:pwc_rhs}, and also applying the first order boundary condition \eqt{eq:bc_out} for both out-going and in-coming directions for the boundary term ($*$), we get:
\begin{align}
\text{LHS} :=&  -\cancelto{\text{II}}{\oint Q\;b_i\;\vo\cdot\vn dA} + \cancelto{\text{I}}{\oint\sigma_a\psi b_i \vo\cdot\vn dA} + \int (\vo\cdot\grad\psi)(\vo\cdot\grad b_i) dV \nonumber\\
             &  -\cancelto{\text{I}}{\oint \sigma_a\;\psi b_i\;\vo\cdot\vn dA} + \boxed{\int\sigma_a\div(\vo\psi b_i) dV} \nonumber\\
             &  +\int\sigma_a^2\psi b_i dV \\
\text{RHS} :=&  -\cancelto{\text{II}}{\oint b_iQ\vo\cdot\vn dA} + \int Q\vo\cdot\grad b_i dV  \nonumber\\
             &  + \int\sigma_a\;Q\;b_i\;dV
\end{align}
We can see that the ``I'' terms cancels out each other, and the same for the ``II'' terms. Finally, expand out the boxed term in LHS:
\be
\boxed{\int\sigma_a\div(\vo\psi b_i) dV} = \int\sigma_a b_i\vo\cdot\grad\psi dV + \int\sigma_a\psi\vo\cdot\grad b_i
\ee
Our final form becomes:
\begin{align}
\text{LHS} :=&  \int (\vo\cdot\grad\psi)(\vo\cdot\grad b_i) dV \nonumber\\
             &  \int\sigma_a b_i\vo\cdot\grad\psi dV + \int\sigma_a\psi\vo\cdot\grad b_i \nonumber\\
             &  +\int\sigma_a^2\psi b_i dV \\
\text{RHS} :=&  \int Q\vo\cdot\grad b_i dV  \nonumber\\
             &  + \int\sigma_a\;Q\;b_i\;dV
\end{align}
The same as Dr. Ragusa's derivation. However, in my implementation I only used the first order boundary condition for the out-going directions, as explained in the Boundary condition section. Therefore, the ``I'' and ``II'' terms doesn't cancel themselves out completely. The in-coming portion of those terms remains.

%%%%%%%%%%%%%%%%%%%%%%%%%%%%%%%%%%%%%%%%%%%%%%%%%%%%%%%%%%%%%%%%%%%%%
\end{document}
%%%%%%%%%%%%%%%%%%%%%%%%%%%%%%%%%%%%%%%%%%%%%%%%%%%%%%%%%%%%%%%%%%%%%
%%%%%%%%%%%%%%%%%%%%%%%%%%%%%%%%%%%%%%%%%%%%%%%%%%%%%%%%%%%%%%%%%%%%%

